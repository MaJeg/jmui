\section{Reproduction of human interaction properties}

%We presented in the previous sections examples of theoretical interactions between two agents. These simulations are rather theoretical and were not created to reproduce the signals variations, transitions or conflict durations between participants. Having in mind our final goal of having an agent able to ensure an effective coordination with a user in real-time, we want to ensure that our agent is able to reproduce the behaviors observed in human 

%Nous avons présenté dans les chapitres 6 et 7 des exemples de situations d’interaction
%entre deux agents vérifiant les sept hypothèses présentées en section 6.1 du
%chapitre 6. Nous rappelons ces hypothèses dans le tableau 9.1.
%Les simulations présentées dans ces chapitres sont théoriques, ne reproduisant
%en rien les variations de signaux, les durées de transition et de conflit observées dans
%les interactions humaines. Nous souhaitons maintenant aller vers une interaction
%utilisateur-agent où les deux participants s’échangent le tour en faisant varier leur
%volume sonore et leur hauteur de voix. Pour aller vers une interaction naturelle
%où l’utilisateur et l’agent coordonnent leurs échanges de tour, nous devons nous
%assurer que l’agent soit capable de reproduire les comportements observés au cours
%d’interactions humaines. Pour une interaction réussie avec un utilisateur, l’agent
%et l’utilisateur doivent moduler leurs signaux de sorte de respecter les durées de
%transition et de conflit que l’on peut observer dans des interactions humaines. Cela
%implique que l’agent ait la capacité à percevoir et à réagir aux variations de signaux
%de la même manière qu’un participant humain le ferait, et la capacité à faire varier
%ses signaux de sorte que l’utilisateur interprète correctement ses motivations.