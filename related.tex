
\section{Background and motivations}
\label{backgd}

\subsection{Definition of turn-taking in conversation}

In most human conversations, participants
alternatively speak so as to avoid speaking at the same
time \citep{sacks_simplest_1974}. This phenomenon is often mentionned under
the term turn-taking \citep{sacks_simplest_1974}, and the interval when
the speech is attributed to one participant is called a
turn. 

One of the most noticeable properties of turn-taking in human conversation is that participants often exchange turns
smoothly \citep{heldner_pauses_2010}, the duration of silence between two turns
is very often less than one second, with some situations
when the next speaker begins to speak a few hundreds
of milliseconds before the end of turn of the previous
speaker. It should be noticed that this principle
of smooth exchange of speaking turns varies across
cultures, languages and types of conversation \citep{oconnell_turntaking_2008,stivers_universals_2009},
nevertheless this principle can be applied
to French conversations \citep{mondada_multimodal_2007}, which is the type of
conversation towards which our model applies.
This property of smoothness of the conversation still continue to question researchers interested in the study of human conversations. To explain how human participants in conversations are able to coordinate smoothly their speaking turns, several models have been elaborated. In the next section we review these different models.

\subsection{How human participants exchange turns ?}
\label{social_psychology}

The most well known model explaining how participants are able to coordinate their speaking turns is the model elaborated by Sacks et al. \citep{sacks_simplest_1974}. The authors \citep{sacks_simplest_1974} consider that it exists a coordination
mechanism between participants that permits
the smooth transitions observed during conversations.
More precisely, speaking turns are composed of
Turn Constructional Units (TCU) : group of words or several sentences that the listeners perceive and use to precisely identify moments appropriate to initiate a turn transition, moments known as Transition Relevant Places (TRP).

For Sacks et al. \citep{sacks_simplest_1974}, at these TRP, any deviation of this set of rules is considered
as a violation of the turn-taking system, and a
repair mechanism should be applied to resolve this situation.
Such violations of the turn-taking mechanism
encompass listeners that try to take the turn outside a
TRP leading to competitive overlaps \citep{schegloff_overlapping_2000}.
This first approach is based on the assumption that
listeners are able to predict precisely the current speaker's end of turn, based on the multimodal behavior
of the latter \citep{de_ruiter_projecting_2006,french_turn-competitive_1983,ford_interactional_1996,mondada_multimodal_2007}. 

While being the most widely used and cited approach, other competing approaches have challenged the model elaborated by Sacks et al. \citep{sacks_simplest_1974}.
As an alternative to the prediction approach, the signal reaction approach, initiated by Duncan \citep{duncan_signals_1972} does
not follow this assumption of prediction made by Sacks et al., nor assumes
the existence of the rule system at TRP elaborated by Sacks et al. \citep{sacks_simplest_1974}. The authors following this approach consider
that the coordination of speaking turns is mainly
the result of a negotiation process where both speaker
and listeners actively act towards taking the turn. The
coordination of turns among the participants is based
on verbal and nonverbal signals produced at the very
end of turns. Such signals are produced by the current
speaker and the current listeners to inform the participants
of their willingness, respectively, to give or to take
the turn. The current listener is then free to react to the
end of turn signals by taking or not the turn and the
current speaker is free to react to turn claiming
signals by giving up or continuing the turn. Several turn ending, turn claiming and turn keeping signals have been identified since the seminal article of Duncan \citep{duncan_signals_1972}. These signals encompass for example variations in prosody \citep{duncan_signals_1972,gravano_turn-taking_2011,hjalmarsson_additive_2011}, gesture \citep{duncan_signals_1972,mondada_multimodal_2007}, gaze \citep{kendon_functions_1967,novick_coordinating_1996,oertel_gazeTT_2013} and vocal patterns such as fillers \citep{benus_pragmatic_2011} and inbreath \citep{torreira_planning_2015}. 

Whether participants project the end of turns of the
current speaker, or just react to the occurrence of signals
is highly debated in the turn-taking community.
Some authors argue that, due to the short duration
of transitions, it is impossible for participants to only
react to signals and there must be a projection mechanism
that permits to anticipate the end of turns \citep{magyari_prediction_2012,riest_anticipation_2015},
the reaction to behavioral cues being a kind of back-up
mechanism when the end of turn is hardly
projectable \citep{grosjean_using_1996}. Other authors point out the lack of evidence
showing that the projection mechanism is used
by the participants, and argue that turn transitions are
not optimal as postulated by authors of the projection
approach, permitting to a reaction mechanism to take
place \citep{heldner_pauses_2010}. 

Recent work in psychology have criticized
the assumption of the two preceding approaches, claiming that turn-taking is an affair of ``rights" and ``obligations", particularly related to turn-taking. \citep{oconnell_turn-taking_1990} argued that `` the ultimate criterion for success in conversation is not the smooth interchange of speaking
turns [...] but the fulfillment of the purpose entertained
by participants". According to their view, turn-taking
doesn't exist per se as an independant procedure deliberately
taken into account by participants, but the way
participants exchange turns is grounded in the situational
and environmental context of the conversation.
Thus cultural norms, the purpose of the conversation
and the content of the conversation, among other variables,
directly influence the way participants coordinate
their turns, leading to more or less simultaneous speech
or more or less silence between turns. Several authors
have also shown that interruptions, situations where listeners
take the turn while the speaker didn't finish his
utterance, are not always conflictual, some are, in fact,
cooperative, linked for example to a listener collaboratively
completing the utterance of the speaker \citep{clancy_co-constructed_2015}, and
are not always linked to display of dominance or power
\citep{goldberg_interrupting_1990}.

That means that creating an agent able to vary its
own behavior related to turn-taking, such as choosing to
interrupt the user if it has something important to say,
or letting on purpose a certain amount of silence before
taking the turn, would not degrade the interaction, but,
if it is made coherently with the dialog context, on the
opposite, enrich the interaction. This view has been
supported by recent studies made by \citep{ter_maat_how_2010} and \citep{cafaro_effects_2016} measuring
the effects of different turn-taking behaviors (referred as ``strategic" view) on
the user's impression about the agent.
\citep{ter_maat_how_2010} showed that varying the response time of the utterance of an agent following the end of turn of the user gave different impressions about the personality and the attitude of the agent towards the user, an agent starting systematically a few hundred milliseconds before the end of the user's turn being perceived as more dominant and assertive, contrarily an agent taking the turn seconds after the end of the user's turn being less assertive and more submissive.  
\citep{cafaro_effects_2016} studied the effect of different types of interrupting
behavior (competitive or collaborative) on the user's
impression about the engagement, interpersonal attitude
and involvement of agents. They found that the type of interruption, more than
the disruptive or cooperative strategy employed by the
agent, influenced the user's impression about the interpersonal
attitudes of the participants.

Finally, some approaches considers the turn-taking process as a dynamic process, changing throughout the conversation linked to synchrony processes between participants. Levitan et al. \citep{levitan_entrainment_2015} observed for example some
alignment effects in the amount of silence before participants
take the turn, and showed that the pitch level of
the next speaker at the beginning of the turn tended to
be similar to the prosody level of the previous speaker
at the end of his turn. 
Alignment effects are also observed in respiratory patterns of participants close to the current speaker's end of turn \citep{mcfarland_respiratory_2001}.
Finally, McFarland \citep{mcfarland_respiratory_2001} and Wilson and Zimmerman\citep{wilson_structure_1986} in English and Bailly et al. \citep{bailly_pauses_2012} in French, observed that inter-turns transitions were not randomly distributed but were multiple of a unit of time. According to \citep{wilson_oscillator_2005} this could be explained by an alignment in the cycle of pronunciation of syllables between
the previous speaker and the next speaker.


\subsection{Computational models of turn-taking for embodied conversational agents}
\label{comp_modelling}


The main goal of computational models of turn-taking for user-agent interaction is to satisfy the strict sequencing of speaking turns between users and agents, mostly in
dyadic settings, by accurately detecting when the user has finished to speak while differentiating pauses from end of turn. Accurately recognizing user's ends of turn
will decrease the involuntary overlaps in conversations of the agent linked to bad perceptions of the user's end of turns and will make silence durations between the user's end of turn and the agent's beginning of turn as short as observed
in human conversations \citep{balentine_debouncing_1997,ward_root_2005,raux_optimizing_2012,
jonsdottir_distributed_2013}.

The simplest approaches use fixed temporal threshold to determinate if a silence can be considered as the end of the user's turn. These approaches were proven to be inefficient, decreasing the
smoothness of interaction between user and agent \citep{ward_root_2005}. To
overcome these limits, some authors used a mechanism
of dynamic threshold to make the agent adapt to the
situation and to the user \citep{bohus_decisions_2011,witt_modeling_2014}. In order to improve the recongition accuracy of the different user's actions linked to turn-taking (turn-yielding, taking, grabbing), authors, following the Signal Reaction Approach \citep{duncan_signals_1972}, tried to perceive different signals displayed by the user to inform his current behavior linked to turn-taking.   
This encompass rule-based systems \citep{cassell_embodiment_1999,thorisson_natural_2002}, and probabilistic
models that use either a decision theoretic approach
approach \citep{bohus_decisions_2011,raux_optimizing_2012}, offline machine learning on transcriptions of human conversations \citep{schlangen_from_2006,huang_multimodal_2011} or realtime reinforcement learning \citep{jonsdottir_distributed_2013} to accurately detect the user's end of turns while discriminating end of turns from simple pauses made by the user. 

Contrarily, very few models adopted Sacks et al.'s view that end of turns are primarily projected. As such kind of models, \citep{de_vault_incremental_2011} created a model dedicated to the prediction of the meaning of the user's utterance that could serve in real-time to generate interruptions, or to take the turn without virtually no gap and no overlap.

A second issue more related to barge-in management
have been addressed by \citep{selfridge_continuously_2013}. It relates to observations
made in user-system interactions that dialog systems
often falsely detects user's utterances as barge-ins whereas
the latter is only giving some feedback to the system by
producing a vocal backchannel, or is not speaking to the
system. Several studies have thus explored how a spoken
dialogue system could reliably distinguish barge-in
attempts from cooperative feedbacks or speech not intended
to the system. \citep{selfridge_continuously_2013} tried to improve the perception of barge-in by relying on the results of an incremental speech recognizer : the more confidence given to a particular utterance of the user, the more probable the user produces a barge-in.
\citep{witt_modeling_2014} created a probabilistic response time model based
on statistics of the timings of user response to the system,
including response after the system's end of turn
and barge-in. The model tries to estimate the probability
of the user starting to speak at one moment during
the interaction. \citep{witt_modeling_2014} did not apply his model to barge-in
perception, but he stated that the probability of barge-in
could be used to disambiguate situations where the
system is not sure about the fact that the user is doing a
barge-in. \citep{reidsma_continuous_2011} explored the possibility to use a classifier
to automatically distinguish between user competitive
and cooperative overlaps, based on prosodic, temporal
and some verbal features (mel-cepstral coefficients).
They showed the ability to discriminate in real-time
between listener responses (generic terms designating
both backchannels, assessments, and acknowledgement
which are cooperative) and other dialog moves with a
latency time of 500 ms but with a relatively high error
rate of 26 \%.

Finally several authors designed agents producing turn-taking signals to clarify to the users their intentions related to turn-taking. In multi-party conversations between user and virtual agents or robots, several approaches have shown the effectiveness of controlling the agent's gaze behavior to select the future speaker \citep{mutlu_sotrytelling_2006,bohus_facilitating_2010,al_moubayed_regulating_2015}. In dyadic settings, \citep{skantze_turn-taking_2014} showed that an agent averting its gaze when pausing (but not releasing its turn) helped the user to identify the behavior of the agent.

\subsection{Relating turn-taking behaviors to content and context}

The models presented section \ref{comp_modelling} are essentially dedicated to the detection of the user's behavior related to turn-taking permitting
to reduce moments of silence and overlaps, thus improving the quality of the interaction. They partly rely on the initial assumptions made by researchers on
turn-taking that interrupting or not taking
the turn after the user's end of turn is a violation
of the turn-taking system, and could lead to a ``break
down" in the conversation \citep{cutler_analysis_1986}. Thus in user-agent interactions,
the agent is seen as having ``obligations" towards
taking the turn without leaving a too long silence
moment and not interrupting the user \citep{de_kok_multimodal_2009}.

Contrarily, following \citep{oconnell_turn-taking_1990,clark_using_1996} approaches,  several authors created computational models of
turn-taking for user-agent or agent-agent interactions
that varied the turn-taking behavior of the agent based
on several variables. \citep{selfridge_bidding_2009} created a model for a dyadic
agent-agent setting, where the motivation of the agents
to take the turn was mostly driven by the importance
of the utterance they had to say. To evaluate if they
will take the turn or not, the agents evaluated in parallel
the importance of the utterance it has or it is saying
and the turn-holding or turn-taking cues displayed to
judge if they will take the turn or keep the turn. \citep{lessmann_towards_2004,thorisson_multiparty_2010}
used both similar notions of ``intention to speak" \citep{lessmann_towards_2004} or
``urgency to speak" to drive the behavior of their agent
\citep{thorisson_multiparty_2010}. \citep{ravenet_conversational_2015} created a model where the participants varied
their behavior related to turn-taking based on their interpersonal
attitudes. In order to have a richer and more
natural turn-taking, these models show the importance
of taking into account factors related to personality,
interpersonal attitude, emotion and the content of the
conversation or at least letting the behavior of the agent
vary according to these factors.

\subsection{Continuous interaction paradigm and turn-taking}

These approaches to turn-taking are somewhat rigid, the behavior of the agent remaining constant throughout the interaction. This do not take into account the fact that turn-taking behaviors are dynamic, evolving throughout the interaction, as highlighted by Wilson and Wilson \citep{wilson_oscillators_2005} and Levitan et al. 
 \citep{levitan_entrainmnent_2015}.  
  
Observations made by these authors could be related to the continuous interaction paradigm \citep{reidsma_continuous_2011}. In this paradigm, behavior of the partner should not be evaluated once terminated but the agent continuously perceive the behavior of its partner while the latter is producing it, and adapt on the fly to what it perceive. These kind of approach could better reproduce the dynamics of human decision-making and behavior for processes where timing constitute an essential part of the behavior as for turn-taking.
 Continuous interaction paradigm have been applied recently to the real-time generation \citep{bevacqua_multimodal_2010} and eliciting \citep{buschmeier_when_2014} of backchannels by the agent, but not to turn-taking. 
 Applied to turn-taking, such approach could reproduce behaviors that has not yet been modelled in embodied conversational agents, such as false starts, simultaneous starts or complex patterns of conflict between the agents, such as succession of withdrawal and turn-taking try. 
 Second, having an agent continuously adapt to the behavior of the user make him more reactive to the latter (expliquer pourquoi c'est un avantage)
 
 Suite : aborder lien direct entre perception des signaux de l'utilisateur et adaptation temps réel des signaux de l'agent
   
% 
% Secondly, what these authors show is a 
%These works show that turn-taking behavior is not only an affair of social rules to follow by participants \citep{sacks_simplest_1974}, or strategies employed by the participant to show their attitudes towards the other \citep{ter_maat_how_2010} but is also linked to low level sensory-motor couplings, where participants directly perceive and adapt their own signal production to the signals produced by their partners. 
% 
% What are the benefits 

% Manque quelque chose ici !!!
% Arguments : 
%	Aspects à prendre en compte : 



%In this article, we propose a computational model for real-time user-agent interactions that address the question of turn-taking as an emergent property of the
%sensory-motor coupling that exists in human interactions
%and potentially in user-agent interactions. We do
%not say that projection does not exist nor that some end
%of turns could not be detected by the occurrence of discrete
%events, such as stereotyped expressions \citep{duncan_signals_1972,gravano_turn-taking_2011} at
%the end of the previous speaker's turn. We mostly agree
%with \citep{thorisson_modeling_2008} view, that behaviors related to conversations
%and turn-taking are due to several cognitive processes,
%from high level deliberative processes to very low-level
%purely reactive processes. We mostly explore here one
%of the numerous processes that occur in conversations,
%that is a purely reactive process.
%We propose a generative model of turn-taking motivated
%both by cognitive psychology models and by studies
%that have measured alignment or synchrony effects
%without postulating the nature of the cognitive process
%behind these phenomena. This model is based on
%general cognitive psychology theory that describes the
%processes behind the continuous sensori-motor coupling
%existing between participants in interactions and that
%drive the agent's behavior. Several existing models have
%been proposed to that purpose, the majority following
%an embedded approach of cognition. Such approach encompasses
%the non-symbolic nature of cognition. Rather
%than planning entire sequences of actions based on internal
%models, the actions are continuously modulated
%by the agents according to their current goals and directly
%in
%uenced by the information provided by the environment
%(either the physical environment or the other
%participants). The agents is thus engaged in a continuous
%perception-action cycle with the environment? The
%variations in the production of the actions impact the
%environment that changes, and, in return, the modi-
%cations of the state of the environment modify the nature
%of the information received by the agent, the latter
%varying the production of his actions based on the new
%informations provided by the environment. It is thus
%said that the agent's behavior is an emergent property
%of the agent-environment interactions.

%\subsection{Dynamical approaches to cognition}
%
%Several approaches have applied these principles to
%human interactions, such as coordination of oscillating
%members by \citep{haken_theoretical_1985}, locomotion of pedestrians trying to
%avoid each other \citep{rio_follow_2014} or in human conversations, coordination
%of postures \citep{fowler_language_2008} and coordination of gaze \citep{richardson_art_2007}.
%\citep{rio_follow_2014} used the behavioral dynamics theory as conceptualized
%by \citep{warren_dynamics_2006} in their computational model of the coordinated
%motions of pedestrians. The behavioral dynamics
%framework formulates the control of action by a set of
%dynamical systems, taking the form of differential equations
%with control variables that vary in real-time due
%to the variation of the information provided by the environment.
%By precisely dening behavior as a set of
%dynamical system, it provides strong principles for the
%implementation of computational models, potentially
%bridging the gap between the theoretical framework of
%embedded cognition and the issues that are specic to
%the design of computational models. It is worth notice
%that participants coordinating speaking turns are engaged
%in prediction processes in the sense given by \citep{warren_dynamics_2006} :
%the goals of their partners, here either taking or yielding
%the turn, is not an information directly perceived
%by participants.
%The agents have to continuously make associations
%between the multimodal signals produced their partners
%and their current behavior, either participants are
%taking, yielding the turn or not. This association has to
%be made in a potentially noisy environment, where information
%is partly uncertain, and dynamic, and where
%the behavior of the partner is continuously varying. The
%agents are involved in a perceptive task, based on the
%variations of the signal produced by their partners. Being
%the current speaker, the agent has to discriminate
%between two alternatives: is the current listener about
%to take the take the turn or not? Symmetrically, being
%the current listener, the agents has to determine
%whether the speaker is about to yield or to keep the
%floor. The agent's decision is based on the variation of
%the signals produced by the other agents during the interaction
%that the agent can more or less likely perceive.
%These conditions of uncertainty and dynamic choice
%between two alternatives follow the Two Alternative
%Forced Choice task paradigm (TAFC). This paradigm is
%extensively used in cognitive psychology to account for
%the dynamics of perceptual decision-making of human
%agents, which have to discriminate the nature of the
%stimuli they have in front of them top make a decision
%\citep{bogacz_physics_2006}. These models have been applied both to tasks where
%the environment remained the same during the entire
%process of decision-making and tasks where the environment
%varied. \citep{lepora_embodied_2015} applied this paradigm to the study of
%the action dynamics of human participants that were
%asked to select an answer to a question by moving the
%cursor of their mouse to the button corresponding to
%the right answer. They then tried to reproduce the trajectory
%of the mouse cursor by coupling a model of
%perceptual decision-making, the Drift-Diffusion model
%(DDM), to the models that controlled the action of the agent. They have tested several approaches, by changing
%the way the perceptual and action components were
%coupled: either the agent is engaged in the perceptual
%process before acting, and then decide to act when he
%is sure about the nature of the stimuli he perceives, or
%the agent moves his mouse cursor even if he is uncertain
%about the nature of the information he perceives
%and modulates the trajectory of his mouse cursor as he
%becomes more and more certain about the alternative
%to chose. They showed that a parallel model of decision
%and action reproduced better the trajectories of human
%participants.

\subsection{Positioning}

We postulate that this principle of perception and
action in parallel is put in action for the coordination
of turns in human conversations. We assume that the
agents are continuously perceiving the multimodal signals
produced by their partners, and that they modulate
in parallel their own actions (verbal and nonverbal
productions), according to the degree of certainty they
have towards the nature of the behavior of their partners.
In turn, by modulating their own productions, the
agents can early produce signals that are perceived by
their partners and that can push the latter to modulate
their own signals, according to their current goal
(taking or yielding the turn). This strong sensorimotor
coupling may provoke the beginning of turn, or the end
of turn, earlier than if the agent would simply passively
interpreted the signals of its partner without acting in
parallel. What we described here is no more less than a
tight coordination that takes place between the agents,
that could partly explain the smooth transitions observed
in human interactions. We present our model in
the next section and show how it reproduces the properties
of human spoken interactions.